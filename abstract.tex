This report aims to answer one main question and one minor question regarding the galactic aspect of the Milky Way. How is the Milky Way structured, and why are galaxies structured in the way that they are? To answer the first question we use a radio telescope which detects hydrogen. Hydrogen is the most common form of matter in the known universe, and by using the aforementioned telescope and calculating the Doppler effect backwards, we can estimate how much hydrogen there is at every given point in the sky. We use galactic coordinates to determine the position of ''clouds'' of hydrogen, and after we process the data through several steps we can also place the ''clouds'' on the galactic scale, and therefore map our galaxy. We also explore the current explanations as to why galaxies assume certain structures by presenting current scientific theories regarding the matter.