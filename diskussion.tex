\chapter{Diskussion}
Resultatet vi fått verkar stämma överens med tidigare undersökningar, givet den spiralarm som vår karta avslöjar. Vintergatans struktur är numera allmän kunskap, och den kunskapen ligger i linje med de resultat detta arbete producerat\autocite{chalmers:samma_arbete}.

En brist med kartan vi ritade upp (figur \ref{fig:map_our}) är att vi förkastade mätvärden som gav upphov till två positiva lösningar på $r$. För att undvika detta kan man göra fler observationer vid samma longitud men variera latituden (man sveper med radioteleskopet ''upp och ner'' över det galaktiska planet). Om molnet befinner sig nära oss förväntas det synas vid latituder längre från $0\degree$ jämfört med om det är längre bort. Detta ansåg vi vara utanför vårt arbetes omfång men är något som framtida arbeten kan belysa vidare.

Denna del av rapporten kommer undersöka den andra delen av frågeställningen, nämligen varför galaktiska strukturer uppstår. De två ledande teorierna är ''Density Wave Theory'' och ''SSPSF''-modellen.

%density wave
Grundtanken med Density Wave Theory\autocite{swin.edu:density-wave} är att det i galaxer finns områden i omlopp kring ett masscentrum, i Vintergatans fall ett stort svart hål. Dessa områden har högre densitet, och i dessa områden trycks vätgas ihop vilket påbörjar processen som leder till att stjärnor uppstår. Dessa unga stjärnor lyser starkt vilket vi ser som ljusstarka spiralarmar. I dessa områden saktas även redan etablerade himlakroppar ner, vilket ger upphov till den förhöjda mängd stjärnor som återfinns där. 

På sjuttiotalet fann forskare att denna modell kunde appliceras på Saturnus ringar med förhållandevis lyckade resultat\autocite{wikipedia:density-wave}. Man fann likheter mellan förhållandet mellan Saturnus och dess ringar och förhållandet mellan vintergatans masscentrum och dess spiralarmar. Man behövde anpassa sambanden lite för att de ska passa den mycket mindre skalan, men likheterna styrker definitivt teorin.

%sspsf
En annan förelsagen förklaring är SSPSF-modellen \autocite{caltech:sspsf} som föreslår att galaxers struktur uppstår av chockvågor som uppstår av olika anledningar. Till skillnad mot Density Wave Theory går denna modell att applicera på alla former av galaxer.

Enligt SSPSF-modellen färdas ''tryckvågor'' genom galaxen som produceras av bland annat solvindar och supernovor. Dessa tryckvågor leder till att vätgas med större sannolikhet än vanligt trycks ihop och bildar nya stjärnor. (Stjärnor bildas när vätgas trycks ihop så pass mycket att dess egna gravitation drar till sig mer vätgas som ökar trycket så pass mycket att fusion uppstår).\autocite{nasa:star_formation} Tryckvågorna lämnar då bakom sig en högre koncentration av nya stjärnor, och eftersom nya stjärnor lyser starkare än äldre stjärnor är det möjligt att  urskilja diverse armstrukturer ur galaxer.

%jämför sspsf och density wave
Man kan konstatera att båda modellerna förlitar sig på ett område med förhållandevis högre täthet för att förklara stjärnornas strukturella formation. Density Wave Theory menar att områden med högre täthet färdas i omlopp runt en himmlakropp och att andra himlakroppar rör sig långsammare genom dessa områden. SSPSF-modellen menar att formationen uppstår på grund av tryckvågor - vilka genom sin natur medför högre täthet - som färdas genom universum. Skillnaden här är att SSPSF-modellens tryckvågor inte är bundna till ett omlopp, utan rör sig någorlunda fritt ur ett astronomiskt perpektiv, till skillnad mot de områden som Density Wave Theory beskriver.